%----------------------------------------------------------------
%
%  File    :  chapter5.tex
%
%  Authors :  Keith Andrews, IICM, TU Graz, Austria
%             Manuel Koschuch, FH Campus Wien, Austria
% 
%  Created :  22 Feb 96
% 
%  Changed :  30 Oct 2008
%  !TEX root = ./thesis.tex
%----------------------------------------------------------------


\chapter{Fazit}
\label{chap:xamarinformsconclusion}

	Für die Entwicklung von Software auf den vollen Funktionsumfang eines Frameworks zurückgreifen zu können ermöglicht es Probleme gezielt zu lösen. Dabei ist es verständlich, dass ein Framework, wie beispielsweise Xamarin.Forms, nur einen Teil des Funktionsumfanges der Zielplattform zur Verfügung stellt. Kreativität und Erfahrung in der Software/Applikationsentwicklung spielen hierbei ein große Rolle um das Framework so zu verwenden, dass die Software entsprechend der Anforderungen entwickelt wird.

	Analyse und Design waren vor Beginn und während der Entwicklung der MCKB App ständig präsent und haben direkten Einfluss eingenommen. Wurde zum Beispiel das iOS Projekt exakt der Vorgaben in dem Design File angezeigt, mussten immer wieder kleine Änderungen an der Android Version vorgenommen werden. Jedoch musste dabei kein Android spezifischer Code geschrieben werden, sondern durch plattformspezifische Parametrierung jene Anpassung vorgenommen werden, um ein ähnlich strukturiertes Design, wie in der iOS Version, zu erzielen. Anfängliche Stolpersteine, die nach wenigen Zeilen Code aufgetreten sind, konnten sowohl durch Recherche zu den Besonderheiten von Xamarin.Forms als auch Kreativität, aus dem Weg geräumt werden.

	Der Umstand, dass Xamarin.Forms den großen Vorteil hat, wenig Code doppelt oder dreifach zu schreiben, steht dessen eingeschränkten Funktionsumfang für plattformspezifische Funktionen gegenüber. Ist man es gewohnt Probleme die eventuell nur Android betreffen in dessen Projekt statt der PCL zu beheben, so ist dies zwar möglich, kann im Verlauf jedoch zu ungewollten Nebeneffekten im weiteren Verlauf der Programmierung führen. Speziell bei der Entwicklung der MCKB App kam es immer wieder vor, dass ein spezifisches Problem zuerst im Projekt der Zielplattform behoben werden konnte, jedoch beim Aufkommen eines ähnlichen Problems, das Design zur Funktion die jenes Problem lösen sollte, überdacht werden musste.

	Ein direkter Vergleich der Funktionen der Android App mit der CP App zeigt, dass das immerwährende Überdenken des Designs dazu geführt hat, dass die Applikationen laufend evaluiert und validiert worden sind. Jedoch spielte die erste CP App (STM32CP) eine essenzielle Rolle, weil bei der Entwicklung jener App viele Erfahrungen gesammelt werden konnten.
