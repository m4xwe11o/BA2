%----------------------------------------------------------------
%
%  File    :  chapter5.tex
%
%  Authors :  Keith Andrews, IICM, TU Graz, Austria
%             Manuel Koschuch, FH Campus Wien, Austria
% 
%  Created :  22 Feb 96
% 
%  Changed :  30 Oct 2008
%  !TEX root = ./thesis.tex
%----------------------------------------------------------------


\chapter{Fazit}
\label{chap:xamarinformsconclusion}

	Für die Programmierung von Software auf den Vollen Funktionsumfang eines Frameworks zurückgreifen zu können ermöglicht es Probleme gezielt zu lösen. Dabei leuchtet es ein, dass ein Framework wie beispielsweise Xamarin.Forms nur einen Teil des Funktionsumfanges der Zielplattform zur Verfügung stellt. Kreativität und Erfahrung in der Software/Applikationsentwicklung spielen hierbei ein Große Rolle das Framework so zu verwenden um die Software Anforderungen umzusetzen.

	Analyse und Design waren vor Beginn und während der Entwicklung der MCKB CP App ständig präsent und haben direkten Einfluss eingenommen. Wurde zum Beispiel das iOS Projekt exakt der Vorgaben in dem Design File angezeigt, mussten immer wieder kleine Änderungen an der Android Version vorgenommen werden. Jedoch musste dabei kein Android spezifischer Code geschrieben werden, sondern durch plattformspezifische Parametrierung jene Anpassung vorgenommen werden um ein ähnlich Strukturiertes Design wie in der iOS Version zu erzielen. Anfängliche Stolpersteine die nach wenigen Zeilen Code aufgetreten sind konnten sowohl durch Recherche zu den Besonderheiten von Xamarin.Forms als auch Kreativität, aus dem Weg geräumt werden.

	Der Umstand das Xamarin.Forms den großen Vorteil hat wenig Code doppelt oder dreifach zu schreiben steht dessen Eingeschränkten Funktionsumfang für plattformspezifische Funktionen gegenüber. Ist man es gewohnt Probleme die eventuell nur Android betreffen in dessen Projekt statt der PCL zu beheben so ist dies zwar möglich, kann im Verlauf jedoch zu ungewollten Nebeneffekten im weiteren Verlauf der Programmierung führen. Speziell bei der Entwicklung der MCKB App kam es immer wieder vor das ein spezifisches Problem zuerst im Projekt der Zielplattform behoben werden kann, jedoch beim Aufkommen eines ähnlichen Problems das Design zur Funktion die jenes Problem lösen soll überdacht werden musste.

	Ein direkter Vergleich der Funktionen der Android App mit der CP App zeigt, dass das immerwährende überdenken des Design dazu geführt hat das die Applikation laufend evaluiert worden ist. Und dabei sogar für zwei Zielplattformen zur gleichen Zeit. Jedoch spielte die erste CP App (STM32CP) eine essenzielle Rolle, weil bei der Entwicklung jener App viele Erfahrungen gesammelt werden konnten.

	Bla Bla
