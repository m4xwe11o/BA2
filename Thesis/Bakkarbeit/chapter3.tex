%----------------------------------------------------------------
%
%  File    :  chapter3.tex
%
%  Authors :  Keith Andrews, IICM, TU Graz, Austria
%             Manuel Koschuch, FH Campus Wien, Austria
% 
%  Created :  22 Feb 96
% 
%  Changed :  30 Oct 2008
%  !TEX root = ./thesis.tex
%----------------------------------------------------------------


\chapter{Designentwicklung der MCKB Applikation}
\label{chap:xamarinformsdevelopment}

	Als Basis für diese Arbeit dienen zwei während des Studiums entwickelte mobile Applikationen, nämlich:

	\begin{enumerate}
		\setlength\itemsep{0em}
		\item \textbf{STM32KB}\footnote{Download via http://m4xwe11o.ddns.net/MAD-Test/App/stm32kb.apk} - Mit Android Studio, in JAVA, entwickelte Applikation im 4. Semester an der FH Campus Wien
		\item \textbf{STM32CP} - Mit Visual Studio for Mac, in C\#, entwickelte CP Variante im 5. Semester an der FH Campus Wien
	\end{enumerate}

	Die Erstere Applikation wurde im Zuge der \textbf{Mobile App Development (MAD)} Vorlesung zur Vertiefung von Software Engineering Wissen unter Android Studio in JAVA entwickelt. Diese App ermöglicht die Darstellung von Artikeln zur Programmierung eines Mikrocontroller $\mu$C der Firma STM, um häufige Fragen zur Konfiguration unterschiedlicher Funktionen wie zum Beispiel \textit{UART (Universal Asynchronous Receive and Transmit)}, \textit{ADC (Analogue Digital Converter)} oder \textit{CAN (Controller Area Network)} einfach erklärt nachzulesen. Die App selber stellt die in einer MySQL Datenbank gespeicherten Artikel dar und ermöglicht es angemeldeten Benutzern bestehende Artikel zu editieren und zu speichern.

	Da die STM32KB App selber keine Daten speichert sondern nur darstellt, wurde eine Infrastruktur entworfen, um die für die App notwendigen Daten mobil abrufen zu können. Es wurde unter anderem eine MySQL Datenbank und ein Webserver auf einem Raspberry PI B+ eingerichtet. Neue Artikel konnten über eine PHP Webseite\footnote{http://m4xwe11o.ddns.net/MAD-Test/webwriter.php} in die Datenbank geschrieben werden. Die App lädt die Artikel aus der MySQL Datenbank über verschiedenste PHP Dateien in eine lokale SQL-Lite Datenbank.

	Für die Entwicklung der STM32CP App (geschrieben mit Xamarin.Native) greift auch diese App auf die schon bestehende Infrastruktur an Server und Datenbank zurück, um die spezifizierten funktionalen Anforderungen zu erfüllen.

	\newpage
	Entsprechend der Anforderungen an die STM32KB App, wurden folgende funktionale Anforderungen im dazugehörigen \textit{Software Requirements Specification Document (SRS)} festgehalten:
	\begin{itemize}
		\setlength\itemsep{0em}
		\item \textbf{Autor Login} - Benutzer können sich in der App anmelden und sehen einen EDIT Button bei Artikeln
		\item \textbf{Autor Registrieren} - Anwender können sich registrieren, um als Autoren freigeschaltet zu werden
		\item \textbf{Artikel Lesen} - Benutzer können Artikel lesen
	\end{itemize}

	\begin{figure}[h!]
		\centering
		\begin{subfigure}
			\centering
			\includegraphics[width=.3\textwidth]{images/stm32kb-loginscreen.png}
		\end{subfigure}
		\begin{subfigure}
			\centering
			\includegraphics[width=.3\textwidth]{images/stm32kb-registerscreen.png}
		\end{subfigure}
		\begin{subfigure}
			\centering
			\includegraphics[width=.3\textwidth]{images/stm32kb-articlescreen.png}
		\end{subfigure}
		\caption{Design der STM32KB App - v.l.n.r. Login, Registrierung, Lesen}
		\label{fig:stm32kbApp}
	\end{figure}

	In Abbildung \ref{fig:stm32kbApp} sind jene funktionale Anforderungen mit deren Layouts aus der App dargestellt. Die Menüführung erfolgt großteils über Buttons und der Zurück Funktion von Android über den Pfeil links oben oder jene entsprechenden Taste auf dem Android Gerät.

	Für die Entwicklung beider CP Apps wurde eine andere Art der Menüführung herangezogen. Statt einer Navigation über Buttons und Pfeile, sind die funktionalen Anforderungen in Form von \textit{Tabs (Registerkarten)} umgesetzt worden. Ein Tabbed Layout steht sowohl unter Xamarin.Native, als auch Xamarin.Forms zur Verfügung und wird von dem jeweiligen Zielbetriebssystem Renderer so gerendert, dass es den typischen Design Merkmalen entspricht.

	\newpage
	\begin{figure}[h!]
		\centering
		\begin{subfigure}
			\centering
			\includegraphics[width=.4\textwidth]{images/STM32CP_Droid-final.png}
		\end{subfigure}
		\begin{subfigure}
			\centering
			\includegraphics[width=.4\textwidth]{images/STM32CP_iOS-final.png}
		\end{subfigure}
		\caption{Design der STM32CP App}
		\label{fig:stm32cpApp}
	\end{figure}

	In Abbildung \ref{fig:stm32cpApp} ist das veränderte Menü durch \textit{Tabs} ersichtlich. Dem Anwender wird durch ein übersichtlicheres Design die Handhabung der Applikation vereinfacht.

	Da die STM32CP Applikation mit Xamarin.Native entworfen wurde, musste das Design sowohl für Android s.h. Abbildung \ref{fig:stm32cpApp} links, als auch iOS, rechts in Abbildung \ref{fig:stm32cpApp}, getrennt entworfen werden. Zwar klingt dies nun nach doppeltem Aufwand für das Design, jedoch hielt sich dies in Grenzen. Das Design einer iOS App ist durch das \textit{Storyboard}, welches für iOS Apps verwendet wird, rasch entworfen. Per Drag\&Drop lassen sich die gewünschten UI Elemente im Layout platzieren und müssen durch Anpassung der \textit{Constraints}\footnote{Durch einen Constraint wird eine Beziehung zwischen Rahmen des Layouts und dem UI Element erzwungen, damit dessen relative Position immer gleich ist} positioniert werden. Für das Design der Android Version kann auf die \textit{XML} Dateien der STM32KB App zurückgegriffen werden, sodass mit Copy\&Paste bestimmter Elemente das Design rasch entworfen ist.

	\newpage
	\begin{figure}[h!]
		\centering
		\begin{subfigure}
			\centering
			\includegraphics[width=.4\textwidth]{images/MCKB-Droid-reading-edit.png}
		\end{subfigure}
		\begin{subfigure}
			\centering
			\includegraphics[width=.4\textwidth]{images/MCKB-iOS-reading-edit.png}
		\end{subfigure}
		\caption{Designentwurf der MCKB App - Beispiel Artikel lesen}
		\label{fig:mckbApp}
	\end{figure}

	Vergleicht man dies nun mit dem ersten Design Entwurf der MCKB Applikation, ersichtlich in Abbildung \ref{fig:mckbApp}, so ist trotz systemspezifischer Designunterschiede, eine klare Ähnlichkeit zwischen den Zielplattformen zu erkennen. Die \textit{Page} der App ist zur Gänze in der PCL Klasse umgesetzt worden und benötigt einen minimalen zielplattformspezifischen Code für die Darstellung der UI Elemente, jedoch nicht für die Geschäftslogik. 

	Für Android, siehe Abbildung \ref{fig:mckbApp} links, ist es klassisch das die \textit{Tabs} für ein \textit{Tabbed Layout}\footnote{In früheren Versionen von Xamarin.Forms wurden keine Icons oder benutzerdefinierte Bilder in den Registerkarten angezeigt. In der Version 2.5.0.121934 ist dies bereits möglich.} im oberen Bereich des Bildschirms positioniert sind. Bei iOS hingegen sind die \textit{Tabs}, die eine Page beinhalten, im unteren Bereich des Bildschirms positioniert. Das Wechseln zwischen den \textit{Tabs} funktioniert in der Android Version durch Wischen von rechts nach links, bzw. vice versa, oder auswählen der entsprechenden \textit{Tabs} möglich.
	\newpage

\section{Applikationsspezifizierung}
\label{sec:mckspecs}

	Die Spezifizierungen sollen weitestgehend an die vorhergehenden Applikationen anknüpfen. Ein Schlüsselwort wie \textbf{muss} erzwingt die Umsetzung während \textbf{sollte} die Möglichkeit der Umsetzung vorschlägt. Die Spezifizierungen umfassen folgende Bereiche:

	\begin{itemize}
		\setlength\itemsep{0em}
		\item Design
		\item Funktionale Anforderungen
		\item Nicht funktionale Anforderungen
	\end{itemize}

	\textbf{Design:} Das Design der App muss mit maximal drei Menüebenen bedienbar sein. Sofern Xamarin.Forms ein Tabbed Layout zur Verfügung stellt, so sollte dieses für eine bestmögliche Benutzerfreundlichkeit verwendet werden. Bezüglich der Farbwahl sollte diese harmonisch sein, demnach blasse Farben die nicht zu sehr in den Vordergrund drängen. Für die Auflistung der Artikel ist folgende Hierarchie zu wählen:

	\begin{itemize}
		\setlength\itemsep{0em}
		\item Dropdown für Kategorien
		\begin{itemize}
			\setlength\itemsep{0em}
			\item Kategorie 1 - i.e. \textbf{STM32F1}
			\begin{itemize}
				\setlength\itemsep{0em}
				\item Unterkategorie 1.1 - i.e. \textbf{STM32F1 - UART}
				\item Unterkategorie 1.2 - i.e. \textbf{STM32F1 - CAN}
			\end{itemize}
		\end{itemize}
	\end{itemize}

	\textbf{Funktionale Anforderungen }sollen jene Funktionalitäten beschreiben, die direkt mit den Anforderungen an die Applikation zusammenhängen. Funktionale Anforderungen können nach dem folgenden Schema\footnote{Das SRS Dokument wurde in der Vorlesung Software Engineering im 4. Semester vorgestellt.} definiert werden:

	\begin{table}[h]
		\centering
		\begin{tabular}{ll}
			Use Case                    &  Name der Anforderung\\ \hline
			Kurzbeschreibung            &  Kurze Definition was passieren soll\\ \hline
			Vorbedingung                &  Was muss vorher gegeben sein\\ \hline
			Nachbedingung               &  Wie ist der Zustand nachher\\ \hline
			Fehlersituation             &  Welcher Fehler kann auftreten\\ \hline
			Systemzustand im Fehlerfall &  - \\ \hline
			Akteure                     &  Wer verwendet die Anforderung\\ \hline
			Standardablauf              &  Systematischer Ablauf für die Anforderung\\ \hline
			Alternativabläufe           &  - \\ \hline
			Trigger                     &  Wie wird die Anforderung angestoßen
		\end{tabular}
		\caption{Standard Schema für Funktionale Anforderungen (\textbf{Use Cases}) laut SRS}
		\label{my-label}
	\end{table}

	\newpage
	Die in Tabelle \ref{tab:autorlogin} dargestellte Anforderung musste nicht angepasst werden.

	\begin{table}[h!]
		\centering
		\begin{tabular}{lp{11cm}}
			\textbf{Use Case}        &  \textbf{Autor Login}\\ \hline
			Kurzbeschreibung &  Anwender können sich als Autoren anmelden\\ \hline
			Standard Ablauf  &  
				\begin{tabular}[c]{@{}l@{}}
					1. Der Anwender öffnet die App\\ 
					2. Der Tab \textbf{Login} wird ausgewählt\\
					3. Der Anwender gibt \textit{Username} und \textit{Password} ein \\
					4. Die Applikation übermittelt die eingegebenen Daten an die\\Datenbank und lässt diese überprüfen. \\
					5. Bei erfolgreicher Prüfung ist der User für die Dauer der Ver-\\wendung der Applikation als Autor angemeldet und kann (so-\\fern die Funktion implementiert ist) Artikel erstellen oder be-\\arbeiten
				\end{tabular}\\ 
		\end{tabular}
		\caption{Autoren Login - Funktionale Beschreibung}
		\label{tab:autorlogin}
	\end{table}

	Die in Tabelle \ref{tab:autorregister} beschriebene Anforderung wurde in der ersten CP Version angepasst. In der Nativen Version der Applikation wurden nach Eingabe der Daten vier Fragen an den Anwender gestellt, um dessen Wissen über Mikrocontroller zu prüfen. Die Ergebnisse der Fragen und die Daten aus dem Registrierungsformular wurden anschließend an einen Administrator der App übermittelt, damit dieser den neuen Benutzer freischalten kann. Der Schritt mit den Fragen zur Wissensüberprüfung wurde nicht implementiert, um nur essentielle funktionale Anforderungen zur Verfügung zu stellen.

	\begin{table}[h!]
		\centering
		\begin{tabular}{lp{11cm}}
			\textbf{Use Case}        &  \textbf{Autor registrieren}\\ \hline
			Kurzbeschreibung &  Anwender können sich registrieren um als Autoren freigeschaltet zu werden \\ \hline
			Standard Ablauf  &  
				\begin{tabular}[c]{@{}l@{}}
					1. Der Anwender öffnet die App\\ 
					2. Der Tab \textbf{Registrieren} wird ausgewählt\\
					3. Der Anwender befüllt alle Eingabefelder \\
					4. Der Anwender muss den Hinweis auf Freischaltung durch einen\\ Administrator zustimmen \\
					5. Dem Benutzer wird visuell mitgeteilt das eine Registrierungs-\\anfrage gestellt wurde
				\end{tabular}\\ 
		\end{tabular}
		\caption{Autor registrieren - Funktionale Beschreibung}
		\label{tab:autorregister}
	\end{table}

	\newpage
	Als letzte funktionale Anforderung ist in Tabelle \ref{tab:artikelread} die Anforderung an das Lesen von Artikeln beschrieben. Diese Anforderung ist ein zentraler Baustein für die Implementierung des gemeinsamen Codes für die Geschäftslogik.

	\begin{table}[h!]
		\centering
		\begin{tabular}{lp{11cm}}
			\textbf{Use Case}        &  \textbf{Artikel lesen}\\ \hline
			Kurzbeschreibung &  Benutzer können Artikel lesen \\ \hline
			Standard Ablauf  &  
				\begin{tabular}[c]{@{}l@{}}
					1. Der Anwender öffnet die App\\ 
					2. Der Tab \textbf{Lesen} wird ausgewählt\\
					3a. Sofern neue Artikel zur Verfügung stehen können diese\\durch Bestätigung eines Hinweises herunter geladen werden\\
					3b.	Alternativ werden Artikel beim Start der App\\automatisch geladen\\
					4. In der Page erscheint ein Drop Down Menü zur Auswahl\\ der Kategorie \\
					5. Der Anwender wählt eine Kategorie und gelangt in die Un-\\terkategorie\\
					6. Es werden die Artikel der korrespondierenden Kategorie\\dargestellt
				\end{tabular}\\ 
		\end{tabular}
		\caption{Artikel lesen - Funktionale Beschreibung}
		\label{tab:artikelread}
	\end{table}

	\textbf{Nicht funktionale Anforderungen,} sind Anforderungen an die Qualität in der die Funktionalität zu erbringen ist. Folgende nicht funktionale Anforderungen wurden für die MCKB Applikation definiert:

	\begin{enumerate}
		\setlength\itemsep{0em}
		\item Benutzeroberfläche
		\begin{itemize}
			\setlength\itemsep{0em}
			\item Flache Menüebenen bei Artikelauswahl
			\item Einfache Menüführung
			\item Harmonische Farben
		\end{itemize}
		\item Leistungsanforderungen
		\begin{itemize}
			\item Server Time-Out von maximal 30 Sekunden
			\item Benutzernachrichten i.e. \textit{Toast} oder \textit{UI Alerts} sollen kurz und informativ sein
		\end{itemize}
		\item Entwicklung
		\begin{itemize}
			\item Visual Studio for Mac oder Visual Studio 2015
			\item Android ab Version 7.0
			\item iOS ab Version 11.3
		\end{itemize}
	\end{enumerate}

	\newpage

\section{Anpassungen an die Serverumgebung im Vergleich zur Xamarin.Native Version}
\label{sec:mckbspecs}

	Für die erste CP App konnte über die Xamarin Komponente \textit{MySQL Plugin} über Port 3306 (\textit{MySQL}) auf die Datenbank zugegriffen werden\cite{Maximilian2017}. Auf Kosten der Sicherheit war dies eine einfache Lösung, um Daten aus der Datenbank in die Applikation zu laden. Um diesen Zugriff umsetzen zu können, musste der MySQL Port auf dem Server freigegeben werden, ein \textit{Port-Forwarding} auf dem Router und ein \textit{NAT} auf der Firewall eingerichtet werden.

	Am 18. Mai 2018 wurde der Komponenten Store für Xamarin abgeschaltet\footnote{https://docs.microsoft.com/en-us/xamarin/cross-platform/troubleshooting/component-nuget?tabs=vswin}, um nur noch den \textbf{NuGET} Store für Plugins zu verwenden. Dies hatte zur Folge, dass das Hinzufügen des \textit{MySQL Plugins} in der PCL nicht mehr möglich war. Zwar kann man das Plugin für den direkten Datenbank Zugriff dem Android und iOS Projekt einzeln hinzufügen, jedoch muss dann sowohl für das Android, als auch das iOS Projekt die Geschäftslogik für den Zugriff auf die Datenbank zweimal implementiert werden.

	Infolgedessen folgt die MCKB App den Vorgaben von Microsoft und verwendet eine \textit{RESTFul API}\footnote{https://docs.microsoft.com/en-us/xamarin/cross-platform/data-cloud/web-services/} (REST - Representational State Transfer), um die Daten aus der Datenbank in die App zu laden. Um diesen Zugriff auf dem Webserver über eine Webservice API zu ermöglichen, musste die genannte \textit{RESTFul API} implementiert werden. Die Kommunikation mit dem Server ist in Abbildung \ref{fig:restfulcommunication} dargestellt.

	\begin{figure}[h!]
		\centering
		\includegraphics[width=1\textwidth]{images/restfull-communication.png}
		\caption{Client - Server Kommunikation}
		\label{fig:restfulcommunication}
	\end{figure}

	Die in Abbildung \ref{fig:restfulcommunication} dargestellte Kommunikation wird beispielsweise durch einen \textit{GET Request} mit einer spezifischen URL, zum Beispiel \textit{http://ipaddress:8000/api/users} vom Client ausgelöst. Der Server nimmt die Anfrage entgegen, verarbeitet diese und sendet eine \textit{Response} im \textit{JSON (JavaScript Object Notation)} Format an den Client zurück.

	Die API baut auf \textit{Node JS} auf und benötigt folgende Dateien:

	\begin{lstlisting}[caption={Webservice Aufbau},label={lst:restfulapi},captionpos=b,style=BashInputStyle]
server@BA-SERVER:~/BA-RESTFUL-API$ ls -la
drwxrwxr-x  8 server server 4096 Mai 29 13:04 .git
drwxrwxr-x 55 server server 4096 Mai 29 12:55 node_modules
-rw-rw-r--  1 server server  249 Mai 29 12:55 package.json
-rw-rw-r--  1 server server 3899 Mai 29 13:04 REST.js
-rw-rw-r--  1 server server 1328 Mai 29 00:44 Server.js
	\end{lstlisting}
	
	Die Datei \textit{Server.js}, in Zeile sechs des Codeabschnitts \ref{lst:restfulapi}, startet den Webserver und die Schnittstelle \textit{/api}, um GET oder POST Requests anzunehmen. In der Datei \textit{REST.js} sind die Schnittstellen zum Aufrufen von Daten konfiguriert. Der Ordner \textit{node\_modules} beinhaltet Pakete, die für das Webservice, beispielsweise \textit{mysql}, \textit{md5} (erzeugt Hashwerte für Passwörter) oder \textit{moment} (erzeugt einen Timestamp), benötigt werden.

	Folgende Schnittstellen sind für die App definiert und implementiert\footnote{Einstiegspunkt ist http://m4xwe11o.ddns.net:8000/api.}:

	\begin{itemize}
		\setlength\itemsep{0em}
		\item GET \textbf{/administrators:} Liefert alle in der Datenbank gespeicherten Administratoren
		\item GET \textbf{/administrators/:administrator\_id:} Liefert einen bestimmten Administrator anhand dessen ID zurück
		\item GET \textbf{/articles:} Liefert alle in der Datenbank gespeicherten Artikel
		\item GET \textbf{/articles/:article\_id:} Liefert einen bestimmten Artikel anhand dessen ID zurück
		\item GET \textbf{/article/count:} Liefert die Anzahl an Artikeln zurück
		\item GET \textbf{/users:} Liefert alle in der Datenbank gespeicherten Benutzer
		\item POST \textbf{/users:} ermöglicht das Hinzufügen von neuen Benutzern zur Datenbank
	\end{itemize}

	Die Serverantwort wird bei einem der API Aufrufe als \textit{JSON} zurückgeliefert.

	\begin{lstlisting}[caption={Serverantwort - \textbf{/article/count:} API Call},label={lst:restfulapicallcount},captionpos=b,style=JAVA-Own]
{
    "Error": false,
    "Message": "Success",
    "Articles": 27
}
	\end{lstlisting}

	Um einen neuen Benutzer in der Datenbank anzulegen, wird die URL \textit{http://.../api/users} aufgerufen und die notwendigen Parameter übergeben. 

	\begin{lstlisting}[caption={Anlegen eines neuen Benutzers durch URL Encoding},label={lst:restfulapicallnewuser},captionpos=b,style=JAVA-Own]
POST /api/users HTTP/1.1
Host: m4xwe11o.ddns.net:8000
Content-Type: application/x-www-form-urlencoded
Cache-Control: no-cache
Postman-Token: b544e89d-a5e0-c3ba-db6a-5efb0122e27a

surname=Thomas&firstname=Pessl&address=Musterstrasse+18%2F2%2C+1110+Wien
&email=maximilian%40pessl.at&password=supersafe&confpassword=supersafe
	\end{lstlisting}

	In Codeabschnitt \ref{lst:restfulapicallnewuser} wird ein HTTP POST auf die URL \textbf{/api/users} aufgerufen und in Zeile acht alle notwendigen Parameter als Teil der URL übergeben. Die Geschäftslogik für das Anlegen eines neuen Users ist in Anhang \ref{chap:app} in Abbildung \ref{fig:serversidecode} zu finden.













