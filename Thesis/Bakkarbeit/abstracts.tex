%----------------------------------------------------------------
%
%  File    :  abstracts.tex
%
%  Authors :  Keith Andrews, IICM, TU Graz, Austria
%             Manuel Koschuch, FH Campus Wien, Austria
% 
%  Created :  22 Feb 96
% 
%  Changed :  30 Oct 2008
%  !TEX root = ./thesis.tex
%----------------------------------------------------------------


% --- German and English Abstracts ------------------------------------------------

% --- German Abstract ----------------------------------------------------
\cleardoublepage

\begin{center}
{\Large\bfseries Kurzfassung}
\end{center}

Apple und Android sind heutzutage die größten Vertreter von mobilen Betriebssystemen und erfreuen sich stetig steigender Beliebtheit. Als Software Entwickler muss nach dem Design einer Applikation und der funktionalen Anforderungen schlussendlich eine Entscheidung über die Zielplattform getroffen werden. Diese Wahl gestaltet sich an sich sehr leicht, weil als größte Vertreter entweder Apples iOS oder Googles Android zur Verfügung stehen. Dies spiegelt den klassischen Software Engineering (SE) Prozess mit den Phasen Analyse, Design, Implementierung, Testen und Wartung wieder. Dabei ist es gegenstandslos, ob es sich um die Entwicklung einer Nativen Applikation, oder Cross-Platform (CP) Anwendung handelt. Vor allem in der Implementierung und Test Phase ist die Entwicklung symmetrisch.

Jedoch gibt es bedeutende Unterschiede in den Phasen Analyse und Design. Bei der Verwendung von Cross-Platform Frameworks (CPF) spielt die Analyse der Anforderungen und die darauf anschließende Design Phase eine beträchtliche Rolle. Xamarin.Forms ermöglicht es für die Entwicklung einer mobilen Applikation die Anwendung aus einer sehr objektiven Betrachtungsweise zu spezifizieren. Dabei ermöglicht es dem Entwickler sich auf die Funktionalitäten der Anwendung und nicht auf die zielbetriebssystemspezifischen Design Elemente zu konzentrieren. Die Arbeit befasst sich ausführlicher mit dem CPF Xamarin.Forms und wie sich die Entwicklung einer mobilen Applikation durch Anwendung eines solchen Frameworks verändert.

% --- English Abstract ----------------------------------------------------

\cleardoublepage

%\selectlanguage{english}

\begin{center}
{\Large\bfseries Abstract}
\end{center}

Apple and Android are the largest representatives of mobile operating systems today and are becoming increasingly popular. As a software developer, a decision about the target platform must finally be made after the design of an application and the functional requirements. This choice is very easy in itself because the largest representatives are either Apple's iOS or Google's Android. This reflects the classic Software Engineering (SE) process with the phases analysis, design, implementation, testing and maintenance. It is irrelevant whether it concerns the development of a native application or a Pross-Platform (CP) application. Especially in the implementation and test phases, the development is symmetrical.

However, there are significant differences in the analysis and design phases. When using Cross-Platform Frameworks (CPF), the analysis of the requirements and the subsequent design phase plays a considerable role. Xamarin.forms makes it possible for the development of a mobile application to specify the application from a very objective point of view. It allows the developer to focus on the functionalities of the application and not on the target operating system specific design elements. The work deals in more detail with the CPF Xamarin.forms and how the development of a mobile application changes through the application of such a framework.

%\selectlanguage{austrian}
